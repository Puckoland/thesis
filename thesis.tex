%%%%%%%%%%%%%%%%%%%%%%%%%%%%%%%%%%%%%%%%%%%%%%%%%%%%%%%%%%%%%%%%%%%%
%% I, the copyright holder of this work, release this work into the
%% public domain. This applies worldwide. In some countries this may
%% not be legally possible; if so: I grant anyone the right to use
%% this work for any purpose, without any conditions, unless such
%% conditions are required by law.
%%%%%%%%%%%%%%%%%%%%%%%%%%%%%%%%%%%%%%%%%%%%%%%%%%%%%%%%%%%%%%%%%%%%

\documentclass[
  digital, %% The `digital` option enables the default options for the
           %% digital version of a document. Replace with `printed`
           %% to enable the default options for the printed version
           %% of a document.
%%  color,   %% Uncomment these lines (by removing the %% at the
%%           %% beginning) to use color in the printed version of your
%%           %% document
  oneside, %% The `oneside` option enables one-sided typesetting,
           %% which is preferred if you are only going to submit a
           %% digital version of your thesis. Replace with `twoside`
           %% for double-sided typesetting if you are planning to
           %% also print your thesis. For double-sided typesetting,
           %% use at least 120 g/m² paper to prevent show-through.
  lof,     %% The `lof` option prints the List of Figures. Replace
           %% with `nolof` to hide the List of Figures.
  lot,     %% The `lot` option prints the List of Tables. Replace
           %% with `nolot` to hide the List of Tables.
]{fithesis4}
%% The following section sets up the locales used in the thesis.
\usepackage[resetfonts]{cmap} %% We need to load the T2A font encoding
\usepackage[T1,T2A]{fontenc}  %% to use the Cyrillic fonts with Russian texts.
\usepackage[
  main=english, %% By using `czech` or `slovak` as the main locale
                %% instead of `english`, you can typeset the thesis
                %% in either Czech or Slovak, respectively.
  english, czech, slovak %% The additional keys allow
]{babel}        %% foreign texts to be typeset as follows:
%%
%%   \begin{otherlanguage}{czech}   ... \end{otherlanguage}
%%   \begin{otherlanguage}{slovak}  ... \end{otherlanguage}
%%
%% For non-Latin scripts, it may be necessary to load additional
%% fonts:
%\usepackage{paratype}
%\def\textrussian#1{{\usefont{T2A}{PTSerif-TLF}{m}{rm}#1}}
%%
%% The following section sets up the metadata of the thesis.
\thesissetup{
    date        = \the\year/\the\month/\the\day,
    university  = mu,
    faculty     = fi,
    type        = bc,
    department  = Department of Computer Systems and Communications,
    author      = Filip Karniš,
    gender      = m,
    advisor     = {RNDr. Miloš Liška, Ph.D.},
    title       = {Shongo Reservation System Backend REST API},
    TeXtitle    = {Shongo Reservation System Backend REST API},
    keywords    = {Shongo, REST, API},
    TeXkeywords = {Shongo, REST, API},
    abstract    = {%
      This is the abstract of my thesis, which can

      span multiple paragraphs.
    },
    thanks      = {%
      These are the acknowledgements for my thesis, which can

      span multiple paragraphs.
    },
    bib         = bibliography/example.bib,
    %% Remove the following line to use the JVS 2018 faculty logo.
    facultyLogo = fithesis-fi,
}
\usepackage{makeidx}      %% The `makeidx` package contains
\makeindex                %% helper commands for index typesetting.
% \usepackage[acronym]{glossaries}          %% The `glossaries` package
% \renewcommand*\glspostdescription{\hfill} %% contains helper commands
% \loadglsentries{misc/abbreviations_and_glossaries}  %% for typesetting glossaries
% \makenoidxglossaries                      %% and lists of abbreviations.
%% These additional packages are used within the document:
\usepackage{paralist} %% Compact list environments
\usepackage{amsmath}  %% Mathematics
\usepackage{amsthm}
\usepackage{amsfonts}
\usepackage{url}      %% Hyperlinks
\usepackage{markdown} %% Lightweight markup
\usepackage{listings} %% Source code highlighting
\lstset{
  basicstyle      = \ttfamily,
  identifierstyle = \color{black},
  keywordstyle    = \color{blue},
  keywordstyle    = {[2]\color{cyan}},
  keywordstyle    = {[3]\color{olive}},
  stringstyle     = \color{teal},
  commentstyle    = \itshape\color{magenta},
  breaklines      = true,
}
\usepackage{floatrow} %% Putting captions above tables
\floatsetup[table]{capposition=top}
\usepackage[babel]{csquotes} %% Context-sensitive quotation marks
\usepackage{hyperref}
\usepackage{cleveref}

\begin{document}
%% Uncomment the following lines (by removing the %% at the beginning)
%% and to print out List of Abbreviations and/or Glossary in your
%% document. Titles for these tables can be changed by replacing the
%% titles `Abbreviations` and `Glossary`, respectively.
% \clearpage
% \printnoidxglossary[title={Abbreviations}, type=\acronymtype]
% \printnoidxglossary[title={Glossary}]

%% The \chapter* command can be used to produce unnumbered chapters:
\chapter*{Introduction}
%% Unlike \chapter, \chapter* does not update the headings and does not
%% enter the chapter to the table of contents. I we want correct
%% headings and a table of contents entry, we must add them manually:
\markright{\textsc{Introduction}}
\addcontentsline{toc}{chapter}{Introduction}

Theses are rumoured to be \enquote{the capstones of education}, so
I decided to write one of my own. If all goes well, I will soon
have a diploma under my belt. Wish me luck!

\begin{otherlanguage}{czech}
Říká se, že závěrečné práce jsou \enquote{vyvrcholením studia}
a tak jsem se rozhodl jednu také napsat. Pokud vše půjde podle
plánu, odnesu si na konci semestru diplom. Držte mi palce!
\end{otherlanguage}

\begin{otherlanguage}{slovak}
Hovorí sa, že záverečné práce sú \enquote{vyvrcholením štúdia}
a tak som sa rozhodol jednu tiež napísať. Ak všetko pôjde podľa
plánu, odnesiem si na konci semestra diplom. Držte mi palce!
\end{otherlanguage}

\chapter{Reservation system Shongo}
In this chapter, we will introduce the reservation system Shongo for which this thesis will make \hyperref[rest]{REST API} extension.

Currently, at version v0.9.3, Shongo is a generic system used for managing resources.
Its primary function is to manage resources reservations for virtual meetings and any physical resources, such as physical meeting rooms, vehicles or parking places.
It allows administrators to define available resources (e.g., H.323 MCU or Adobe
Connect Server) and users to reserve the resources mentioned before. \cite{shongo}

CESNET association develops the system as an open-source project available at \href{https://github.com/shongo/shongo}{Github repository}, and it is currently deployed in these domains:
\begin{itemize}
    \item \url{https://meetings.cesnet.cz} - reservation system for H.323/SIP and Adobe Connect virtual rooms within CESNET videoconferencing infrastructure
    \item \url{https://meetings.cesnet.cz/ceitec} - reservation system for physical rooms of CEITEC - Central European Institute of Technology
    \item \url{https://meetings.cesnet.cz/cuni/} - reservation system for users led in authentication system of Charles University \footnote{\url{https://cuni.cz/}}.
\end{itemize}

Users can log in using their profile from any organization belonging to the Czech academic identity federation \footfullcite{eduid}.
After logging in, they can freely use the system and request reservations for available resources.

\enquote{CESNET is an association of universities and the Academy of Sciences of the Czech Republic that operates and develops a national e-infrastructure for science, research and education, including a computer network, computing grids, data repositories, collaborative environments and offering a wide range of services.} \cite{cesnet}

\section{Architecture}
% This section is mostly referenced from another thesis \footfullcite{pavelka2016shongo}, my primary learning material for how the Shongo system works.
I have drawn information for this section from shongo api documentation \footfullcite{shongoapi} and another thesis \footfullcite{pavelka2016shongo}.

The system is composed of a few separate modules. Their interconnections are shown in \Cref{architecture}.
\begin{figure}[!ht]
  \centering
  \caption{Shongo architecture}
%   % generated by Plantuml 1.2018.13      
\definecolor{plantucolor0000}{RGB}{168,0,54}
\definecolor{plantucolor0001}{RGB}{254,254,206}
\definecolor{plantucolor0002}{RGB}{0,0,0}
\begin{tikzpicture}[yscale=-1
,pstyle0/.style={color=plantucolor0000,line width=1.0pt,dash pattern=on 5.0pt off 5.0pt}
,pstyle1/.style={color=plantucolor0000,fill=plantucolor0001,line width=1.5pt}
,pstyle2/.style={color=plantucolor0000,fill=plantucolor0000,line width=1.0pt}
]
\draw[pstyle0] (32pt,38.2969pt) -- (32pt,116.5625pt);
\draw[pstyle0] (94.6593pt,38.2969pt) -- (94.6593pt,116.5625pt);
\draw[pstyle1] (8pt,3pt) rectangle (53.6593pt,33.2969pt);
\node at (15pt,10pt)[below right,color=black]{Bob};
\draw[pstyle1] (8pt,115.5625pt) rectangle (53.6593pt,145.8594pt);
\node at (15pt,122.5625pt)[below right,color=black]{Bob};
\draw[pstyle1] (67.6593pt,3pt) rectangle (119.6357pt,33.2969pt);
\node at (74.6593pt,10pt)[below right,color=black]{Alice};
\draw[pstyle1] (67.6593pt,115.5625pt) rectangle (119.6357pt,145.8594pt);
\node at (74.6593pt,122.5625pt)[below right,color=black]{Alice};
\draw[pstyle2] (83.6475pt,65.4297pt) -- (93.6475pt,69.4297pt) -- (83.6475pt,73.4297pt) -- (87.6475pt,69.4297pt) -- cycle;
\draw[color=plantucolor0000,line width=1.0pt] (32.8296pt,69.4297pt) -- (89.6475pt,69.4297pt);
\node at (39.8296pt,52.2969pt)[below right,color=black]{hello};
\draw[pstyle2] (43.8296pt,94.5625pt) -- (33.8296pt,98.5625pt) -- (43.8296pt,102.5625pt) -- (39.8296pt,98.5625pt) -- cycle;
\draw[color=plantucolor0000,line width=1.0pt,dash pattern=on 2.0pt off 2.0pt] (37.8296pt,98.5625pt) -- (94.6475pt,98.5625pt);
\node at (49.8296pt,81.4297pt)[below right,color=black]{Ok};
\end{tikzpicture}

%   \includegraphics{assets/architecture}
  \makebox[\textwidth]{\includegraphics[width=\paperwidth]{assets/architecture}}
  \label{architecture}
\end{figure}

\subsection{Controller} \label{controller}
The controller module is the heart of the Shongo system. Its primary role is to plan requested reservations (\emph{Scheduler} sub-component) and assign necessary resources to them. Its secondary roles are informing users about their reservations (\emph{NotificationManager} sub-component) and administrate remote virtual resources (\emph{Executor} sub-component) via connected \hyperref[connector]{connectors}. Furthermore, last but not least, save this information to the relational database \emph{PostgreSQL}.

These functionalities are made available for other components with the \emph{Application programming interface} (API) via protocol \emph{XML-RPC}.

\subsection{Connector} \label{connector}
Connector is component that enable users to connect to video rooms with their devices (PC, tablet, smartphone, ...).

It is responsible for establishing, maintaining and monitoring connections with remote virtual resources.
It contains implementations for \emph{Device Agent}s where each \emph{Device Agent} can mediate a specific device (e.g. Adobe Connect, Pexip) functions to the controller using the device's API.
The controller communicates with the connector using the JADE (Java Agent DEvelopment) framework, which enables the connection of multiple connectors to a single controller.

\subsection{Authentication server}
The authentication server authenticates incoming requests for the controller component, authenticates users, finds users from various identity providers and gets information about users for the web client.
Besides that, it also manages the authentication layer of the web conference system Adobe Connect. This layer uses the user's information gained after logging in via system Shibboleth \footnote{Shibboleth, \url{https://www.shibboleth.net/}}.

\subsection{Web client} \label{webclient}
The web client is the system's primary interface for users. Using frameworks \emph{SpringMVC} and \emph{AngularJS}, it generates web pages with data gained from the controller's services.

Thanks to this component, users can authenticate via OpenID Connect \footnote{OpenID Connect, \url{https://openid.net/connect/}} and request a reservation for any available resource comfortably via the web.

\subsection{Command line client}
The command line client is the controller's additional interface used for administration. It enables system administrators to manage resources, user, groups and reservation requests which is not possible using the web client.

\chapter{What to change and why}
% Monolithic systems are out of fashion.
% Nowadays, most of big systems run as micro-services that communicate together but can work as a single part.
% This opens door to containerization which makes the deployment so much easier.
% ALE ONO TO UZ JE ROZDELENE NA SERVICE VDAKA SPRINGU AJ TERAZ.

Second reason is user interface \Cref{webclient}. Web MVC is not as powerful as today's JavaScript frameworks like Angular or React.
So it become the goal of another thesis. \cite{drobnakm}
But since until now the whole communication of system was via Spring Framework in Java some new API has to be created.

\scalegraphics{assets/shongo_architecture}{New Shongo architecture}{architecture_new}

As we can see on \Cref{fig:architecture_new}, Web Client component has been moved out of the Shongo system and is developed and maintained separately at \href{https://github.com/shongo/shongo-frontend}{new Github repository}. This part has been done in another thesis \cite{drobnakm}.
Additionally it is communicating with Controller via REST API instead of XML-RPC. The implementation of this REST API is the subject of this thesis.

\section{API options}
Today there are 3 general approaches for such API.
\subsection{GraphQL}
\subsection{RPC}
\subsection{REST API}

\chapter{Used Technologies} \label{cha:technologies}
This chapter will introduce the technologies used to implement the REST API for the Shongo system.


\section{Representational State Transfer} \label{rest}
Representational State Transfer or \texttt{REST} is an architectural style for distributed hypermedia systems. \citeauthor{fielding2000rest} introduced the REST first time in \citeyear{fielding2000rest} \cite{fielding2000rest}.
\subsection{REST design principles}
Unlike other API designs such as XML-RPC or SOAP, REST APIs support any format for communication. However, rules have to be followed so the API can be called \emph{RESTful}.

These rules are called REST design principles --- also known as architectural constraints \cite{ibmrest}:
\begin{enumerate}
    \item \textbf{Uniform interface} -- All API requests for the same resource should have the same form disregarding the origin of the request. The same piece of data, such as the name or email, belongs to one uniform URI (uniform resource identifier).
    \item \textbf{Client-server decoupling} -- The client and server sides of the application should be completely independent of each other. The client should have only one piece of information — the URI of the requested resource. This rule implies that the client cannot interact with the server in any different ways. Analogically, the server should not affect the client in any other way than passing requested data to it.
    \item \textbf{Statelessness} -- Each request includes all the information needed. In other words, the server is not allowed to store any data related to the request, and there are no server-side sessions.
    \item \textbf{Cacheability} -- As low as possible, requests should be made to reduce traffic and improve performance, so as many as possible resources should be cacheable. The server should also inform the client whether the given data in the response is suitable for caching.
    \item \textbf{Layered system architecture} -- Requests and responses can go through many layers on their way. There may be several different intermediaries in the communication. As a result, neither the server nor the client can assume they are communicating directly with the application or with an intermediary.
    \item \textbf{Code on demand (optional)} -- In some cases, the response can be executable code (Java applets, for example). In these cases, the code should only run on-demand.
\end{enumerate}


\section{Mockoon} \label{sec:mockoon}
Mockoon is an open-source tool for designing and mocking REST APIs.
Because of its mocking capabilities, it was chosen to synchronize REST API and front-end development \cite{drobnakm}.
For this reason, it was also used as an unofficial designing tool for the API.
\scalegraphics{assets/mockoon.png}{Mockoon}{mockoon}


\section{Jackson} \label{sec:jackson}
Jackson\footnote{\url{https://github.com/FasterXML/jackson}}, or \enquote{the Java JSON library}, is an open-source suite of data-processing tools for Java. First of all, it includes \texttt{ObjectMapper}, which can serialize Java objects to JSON (JavaScript Object Notation) and deserialize them back.
It also supports XML (Extensible Markup Language), YAML (YAML Ain't Markup Language), CSV (Comma-Separated Values), and many other formats.
Jackson automatically uses public attributes, or rather getters, to serialize values for JSON and setters to deserialize values obtained from it.


\section{Spring Framework} \label{sec:spring}
The most popular framework for Java applications is Spring, which offers a context for the application.
First of all, the framework starts a container (Spring application context) that manages the application components --- also known as \emph{Beans} --- and wires them together so they can use each other. This wiring is achieved by \emph{dependency injection}.
The components do not create and maintain different features they depend upon; instead, they rely on a separate entity (Spring application) to manage the dependent modules and inject them into components that need them \cite{walls2022spring}.

The Shongo system is written in Java, and the Spring framework features were used for faster and easier development of the system. Its dependency injection feature was used to inject functionalities of common objects, such as services and caches, into other objects that need these functionalities. In addition, Spring’s web features solved the fundamental issues of the \emph{Web Client} implementation, such as HTTP request-response processing and MVC implementation.

The Spring framework also supports REST API implementation, and this work fully utilizes this feature.
Spring uses embedded \texttt{Object\-Mapper} from Jackson library (\Cref{sec:jackson}) internally to serialize and deserialize foreign formats (JSON in most cases) to Java objects automatically. The \texttt{ObjectMapper} is very useful when dealing with HTTP requests which usually use JSON for object formatting. Thanks to this feature, an API developer can easily use simple model objects to acquire or provide data to the client via HTTP communication.
This usage will be discussed further in \Cref{cha:implementation}.


\section{Project Lombok} \label{sec:lombok}
Lombok is an open-source Java library that generates frequently used Java code using simple annotations like \texttt{@Getter}, \texttt{@Setter}, \texttt{@Data} or \texttt{@Builder}.
The entirely equivalent \Cref{lst:vanilla} and \Cref{lst:lombok} are presented below to show an example of what is possible with Lombok~\cite{lombok}.
\begin{lstlisting}[language=Java, caption=Vanilla Java, label=lst:vanilla]
public class DataExample {
  private final String name;
  private int age;
  private double score;
  private String[] tags;
  
  public DataExample(String name) {
    this.name = name;
  }
  
  public String getName() {
    return this.name;
  }
  
  void setAge(int age) {
    this.age = age;
  }
  
  public int getAge() {
    return this.age;
  }
  
  ...
  
  @Override public String toString() {...}
  
  @Override public boolean equals(Object o) {...}
  
  @Override public int hashCode() {...}
\end{lstlisting}
\begin{lstlisting}[language=Java, caption=Java with Lombok, label=lst:lombok]
@Data public class DataExample {
  private final String name;
  @Setter(AccessLevel.PACKAGE) private int age;
  private double score;
  private String[] tags;
}
\end{lstlisting}


\section{Postman} \label{sec:postman}
Postman is an API platform for building and using APIs. It offers an environment and workspaces to test extensive APIs.
For these qualities, the Postman was used in the early stages to test the developed REST API.


\section{OpenAPI Specification} \label{sec:openapi}
The OpenAPI Specification (OAS) defines a standard interface to RESTful APIs, allowing humans and computers to understand the API’s capabilities without access to source code or documentation. When properly defined, a consumer can understand and interact with the remote service with minimal implementation logic.
Documentation generation tools can then use an OpenAPI definition to display the API, code generation tools to generate servers and clients in various programming languages, testing tools, and many other use-cases \cite{openapi}.

The OAS can define endpoints, a data format and schemas for objects passed between server and client, and security specifications that make the access to the API stricter.
The example of OAS is shown in \Cref{lst:openapi}.
\begin{lstlisting}[language=Java, caption=OpenAPI.json, label=lst:openapi]
{
    "openapi": "3.0.1",
    "info": {
        "title": "Shongo definition",
        "version": "v1"
    },
    ...
    "paths": {
        "/api/v1/reservation_requests": {
            "get": {
                ...
            },
            "post": {
                "parameters": [...]
                "responses": [...],
                "requestBody": {...}
            }
        }
    },
    "securitySchemes": {
        "api_key": {
            "type": "apiKey",
            "name": "api_key",
            "in": "header"
        }
    },
    ...
}
\end{lstlisting}

\section{Swagger UI} \label{sec:swagger}
Swagger UI enables anyone to visualize and interact with the API’s resources. It can be automatically generated from OpenAPI Specification, resulting in user-friendly visual documentation \cite{swagger}.
\scalegraphics{assets/swagger}{Swagger UI}{swagger}

\chapter{Implementation and Documentation} \label{cha:implementation}
All classes related to the REST API are stored in the Controller’s package  \texttt{cz.cesnet.shongo.controller.rest}.
Aside from this package, the only changes made are in a few files that retrieve data from the database --- where additional data were required to be loaded --- and inter-domain implementation files\cite{pavelka2016shongo} whose configuration has been merged with the REST configuration.

\section{REST Server}
The cornerstone of the implementation is the REST server which listens for incoming HTTP requests from clients and responds to them in the desired way. The REST server is implemented in the \texttt{RESTApiServer} class.
At the start of the \emph{Controller} service, the REST server is attached to the \emph{Controller}.

\texttt{RESTApiServer} starts a \emph{Jetty}\footnote{\url{https://www.eclipse.org/jetty/}} server and configures it according to the controller configuration. The server then listens on the host and port specified in the configuration.

\section{Configuration}
The REST server configuration is required before implementing API endpoints. These configuration files are stored in the \texttt{config} sub-package.

\subsection{Configuration File Extension}
First, the configuration of the REST server itself is essential. The controller configuration file \texttt{shongo-controller.cfg.xml} has been extended with the parameterization for the new code. Consequently, Shongo administrators can set the REST API server properties inside the \texttt{<rest-api>} XML element. An example of this new configuration is shown in \Cref{conf}. There are several configurable sub elements:
\begin{itemize}
    \item \texttt{<host>} -- the host where the REST API shall serve
    \item \texttt{<port>} -- the port where the REST API shall serve
    \item \texttt{<origin>} -- the allowed origins for the REST API CORS configuration
    \item \texttt{<ssl-key-store>} -- the SSL (Secure Socket Layer) key store from inter-domain extension \cite{pavelka2016shongo}
    \item \texttt{<ssl-key-store-type>} -- the type of key store
    \item \texttt{<ssl-key-store-password>} -- the password for key store
\end{itemize}

\begin{lstlisting}[language=XML, caption=REST configuration example, label=conf]
<?xml version="1.0" encoding="UTF-8" ?>
<configuration>
    ...
    <rest-api>
        <host>meetings.cesnet.cz</host>
        <port>9999</port>
        <origin>http://localhost:4200</origin>
        <origin>https://meetings.cesnet.cz</origin>
        <ssl-key-store>keystore/server.keystore</ssl-key-store>
        <ssl-key-store-password>
            (password)
        </ssl-key-store-password>
    </rest-api>
    ...
</configuration>
\end{lstlisting}

\subsection{Security}
Next, the security and authentication had to be configured. For this purpose, \emph{Spring security} is used. Spring security dependencies have been added to \emph{Controller}'s \texttt{pom.xml}, and security configuration files were stored in \texttt{config.security} sub-package.

The web security is configured in the \texttt{SecurityConfig} class annotated with the Spring \texttt{@EnableWebSecurity} annotation, which allows this class to configure web security\cite{springdocumentation}.
This configuration file defines what should happen to the incoming request. The request filters are added to the request processing, and CORS (Cross-Origin Resource Sharing)\footnote{\url{https://www.w3.org/TR/cors/}} is configured here. The paths that should be skipped in the authentication process are also set here. At the moment, these include OpenAPI, Swagger, inter-domain endpoints\cite{pavelka2016shongo}, and endpoints responsible for processing problem reports.

Additionally, \texttt{AuthFilter} is added, which serves as a REST API middleware. It processes each request (omitting those configured in SecurityConfig) using the following steps:
\begin{enumerate}
    \item Decode an access token wrapped in the HTTP request authorization header as a bearer token.
    \item Check the token authorization.
    \item Convert the token into \texttt{SecurityToken} object which contains an \emph{access token} for Shongo Controller and cached information about the user to whom the token belongs.
    \item Add this new object to the request attributes using the \texttt{TOKEN} key so that subsequent endpoint handling (next middleware or final method) can effortlessly work with it.
\end{enumerate}
If the token is not present or is invalid, the server responds with an HTTP \emph{401 UNAUTHORIZED} response code.

\section{REST Controllers}
This section uses information from Spring documentation \cite{springdocumentation}.
The REST Controllers are responsible for operating the API endpoints and are stored in the \texttt{controllers} sub-package. An example of such a Controller can be observed in \Cref{lst:controller}.

The REST Controllers are annotated with \texttt{@RestController} annotation from the Spring framework, which combines two other annotations:
\begin{description}
    \item \texttt{@Controller} -- makes the class auto-detectable through classpath scanning
    \item \texttt{@ResponseBody} -- binds the return values of the mapped methods to the HTTP response body
\end{description}
This annotation is combined with the \texttt{@RequestMapping} annotation, which uses the web request path to map the incoming request to the method that processes the request and responds to the client. \texttt{@RequestMapping} can also be parameterized, particularly with \texttt{path} (path to the endpoint), \texttt{consumes} (expected incoming body form), and \texttt{produces} (outgoing body form) parameters.

There are usually only few attributes in the controller classes. These attributes are generally Controller’s services, and \emph{dependency injection} is used to access these services. Thanks to Spring annotation \texttt{@Autowired}, they are resolved to \emph{Beans} defined in \texttt{rest-api-servlet.xml} and injected into the controller class. The example of this file can be noticed in \Cref{lst:beans}.

\begin{lstlisting}[language=XML, caption=rest-api-servlet.xml, label=lst:beans]
<?xml version="1.0" encoding="UTF-8"?>
<beans xmlns=
"http://www.springframework.org/schema/beans"
       ...>
    <context:component-scan base-package="cz.cesnet.shongo.controller"/>
    <mvc:annotation-driven />
    <aop:aspectj-autoproxy />
    
    <bean id="controller" class=
    "cz.cesnet.shongo.controller.Controller" factory-method="getInstance"/>
    <bean id="configuration" factory-bean="controller" factory-method="getConfiguration"/>
    <bean id="controllerClient" class=
    "cz.cesnet.shongo.controller.ControllerClient">
        <constructor-arg>
            <bean factory-bean="configuration" factory-method="getRpcUrl"/>
        </constructor-arg>
    </bean>
    <bean id="controllerReservationService" factory-bean="controllerClient" factory-method="getService">
        <constructor-arg value=
        "cz.cesnet.shongo.controller.api.rpc.
        ReservationService" type="java.lang.Class"/>
    </bean>
    ...
    <bean id="cache" class=
    "cz.cesnet.shongo.controller.rest.Cache"/>
</beans>
\end{lstlisting}

The last part of REST Controllers is the declaration and implementation of methods handling distinct REST resources. The methods are annotated with \texttt{@RequestMapping} or rather an exact mapping according to the selected HTTP method (\texttt{@GetMapping}, \texttt{@PostMapping}, \texttt{@PutMapping}, \texttt{@DeleteMapping}).
These annotations also take a path parameter, which is then concatenated to the class’s \texttt{@RequestMapping} path parameter.
In addition, the methods can retrieve HTTP request attributes annotated with \texttt{@RequestAttribute} (most importantly SecurityToken added by \texttt{AuthFilter}), HTTP request parameters annotated with \texttt{@RequestParam}, path parameters annotated with \texttt{@PathVariable} (and specified in the path as \texttt{"\{variableName\}"}) and also the request body annotated with \texttt{@RequestBody}.

\begin{lstlisting}[language=java, caption=ReservationRequestController.java, label=lst:controller]
// Makes this path a Controller (auto-detectable) and binds return values of methods to the HTTP request body
@RestController
// Binds this controller to the path parameter
@RequestMapping("/api/v1/reservation_requests")
public class ReservationRequestController
{
    // Controller service used to manage reservations
    private final ReservationService reservationService;
    
    // @Autowired resolves the service to Bean defined in rest-api-servlet.xml
    public ReservationRequestController(@Autowired ReservationService reservationService)
    {
        this.reservationService = reservationService;
    }
    
    // Binds this method to the HTTP GET request with path defined in @RequestMapping
    @GetMapping
    // Method returns ListResponse<ReservationRequestModel> in response body thanks to @RestController annotation
    ListResponse<ReservationRequestModel> listRequests(
        // Gets security token from request attributes. The token was stored there when AuthFilter processed the request.
        @RequestAttribute(TOKEN) SecurityToken securityToken,
        // Reads optional request parameter allocation_state and stores it in allocationState variable
        @RequestParam(value = "allocation_state", required = false) AllocationState allocationState,
        ...
    )
    {
        // Use Controller service to acquire requested data
        ReservationRequestListRequest request = new ReservationRequestListRequest();
        request.setAllocationState(allocationState);
        ...
        ListResponse<ReservationRequestSummary> response = reservationService.listReservationRequests(request);
        ...
        // Return the acquired data
        return response;
    }
    
    // Binds this method to the HTTP POST request with path defined in @RequestMapping
    @PostMapping
    void createRequest(
        @RequestAttribute(TOKEN) SecurityToken securityToken,
        // Reads the request body and deserializes it to the ReservationRequest object
        @RequestBody ReservationRequest request)
    {
        ...
        // Use Controller service to create reservation according to the requested data
        reservationService.createReservationRequest(
            securityToken, request.toApi());
    }
    
    // Sets possible HTTP responses
    @ApiResponses(value = {
        @ApiResponse(responseCode = "200"),
        @ApiResponse(responseCode = "404", description = "Reservation request not found.", content = @Content),
    })
    // Binds this method to the HTTP DELETE request with path defined in @RequestMapping concatenated with "/{id:.+}". The ':' marks the start of a regex to be used for the id variable (useful to accept only numbers for example). The regex ".+" is used because shongo-id can include URL key characters like ':' or '-'.
    @DeleteMapping("/{id:.+}")
    void deleteRequest(
        @RequestAttribute(TOKEN) SecurityToken securityToken,
        // Reads the id variable defined in request path
        @PathVariable String id)
    {
        // Use Controller service to delete reservation with id defined in path
        reservationService.deleteReservationRequest(
            securityToken, id);
    }
    
    ...
}
\end{lstlisting}

\section{Data Models}
Model classes are mostly POJOs (Plain Old Java Objects) that represent objects received or sent by endpoints. The models are stored in the \texttt{models} sub-package. Usually, attributes with getters and setters are the only content of these classes. Therefore, most of them use \texttt{@Data} annotation from the project Lombok (\Cref{sec:lombok}).
\emph{Jackson} can then serialize and deserialize these classes as described in \Cref{sec:jackson}. Attributes not eligible for serialization are annotated with \texttt{@JacksonIgnore} and for custom name of field in serialized object \texttt{@JacksonProperty("field\_name")} is used.
That does not mean that these classes cannot include anything else. Most importantly, they usually also contain \texttt{fromApi()} and \texttt{toApi()} methods which serve as converters from and to Controller API objects. The \Cref{lst:model} serves as an example of such class.

\begin{lstlisting}[language=java, caption=RoomModel.java, label=lst:model]
// Generates constructor, getters, setters, equals, and hashCode functions
// These generated functions enables Jackson to (de)serialize this object
@Data
// Represents Controller's executable
public class RoomModel {

    // Ignores this attribute during (de)serialization
    @JacksonIgnore
    private Cache cache;

    private String id;
    private ExecutableSummary.Type type;
    private TimeInterval slot;
    private TechnologyModel technology;
    private RoomState state;
    // (De)serializes this attribute to license_count field
    @JsonProperty(license_count)
    private int licenceCount;
    ...

    // Creates this object from data acquired from Controller service
    public static RoomModel fromApi(ExecutableSummary summary)
    {
        RoomModel roomModel = new RoomModel();
        roomModel.setId(summary.getId());
        roomModel.setType(summary.getType());
        roomModel.setSlot(new TimeInterval(summary.getSlot()));
        roomModel.setState(
                RoomState.fromRoomState(
                        summary.getState(), summary.getRoomLicenseCount(),
                        summary.getRoomUsageState())
        );
        ...
        return roomModel;
    }
}
\end{lstlisting}


\section{Error Handling}
Quite many things can go wrong while processing a request. For this reason, the \texttt{error} sub-package was created for files concerning errors and their handling.
For extraordinary cases, new exceptions were made, such as \texttt{UnsupportedApiException} or \texttt{ObjectInaccessible\-Exception}.
These (and any other) exceptions can be thrown during request processing, in which case, the server responds with HTTP response 500 Internal Server Error and prints the \emph{stacktrace} to the HTTP body by default.
To provide better information for the client, \texttt{GlobalController\-ExceptionHandler} class was implemented and annotated with Spring web annotation \texttt{@RestControllerAdvice}.
The methods in the class are annotated with \texttt{@ExceptionHandler}, which takes an exception as a parameter. If this exception is thrown, the request processing intercepts and the annotated method runs instead.
These methods can handle the exception and then respond to the client with the \texttt{ResponseEntity} object. As a result, any object can be passed into the HTTP response body with a desired HTTP response status code.

\begin{lstlisting}[language=java, caption=GlobalControllerExceptionHandler.java, label=lst:err]
@RestControllerAdvice
public class GlobalControllerExceptionHandler
{
    @ExceptionHandler(TodoImplementException.class)
    public ResponseEntity<String> handleTodo(TodoImplementException e) {
        return ErrorModel.createResponseFromException(e, HttpStatus.NOT_IMPLEMENTED);
    }
    ...
}
\end{lstlisting}


\section{Miscellaneous}
The \texttt{ClientWebUrl} class holds the String constants for the endpoints paths. All REST API endpoints paths have the default prefix --- \texttt{/api/v1}.

\texttt{Cache} and \texttt{CacheProvider} classes supply a simple \emph{cache} for several frequently retrieved entities represented by \texttt{ExpirationMap}, which stores the requested entities for a determined amount of time, so the REST API does not have to wait for a much slower database after each request.


\section{Documentation}
The OpenAPI (\Cref{sec:openapi}) specification of the just described implementation of REST API is generated using \texttt{Springdoc} from the Spring framework (\Cref{sec:spring}), so the users can have general and straightforward API documentation available to them.
For this purpose, the \emph{springdoc-openapi} dependency was added to \emph{Controller}'s \texttt{pom.xml}, and the \texttt{OpenApiConfig} class was created. In this configuration file, Springdoc dependencies were imported using Spring's \emph{@Import}, global API attributes were defined with \texttt{@OpenAPIDefinition}, and security was defined with \texttt{@SecurityScheme} annotation.
This is shown in \Cref{lst:openapiconf}.
\begin{lstlisting}[language=Java, caption=OpenApiConfig.java, label=lst:openapiconf]
@Configuration
@EnableWebMvc
@OpenAPIDefinition(
        info = @Info(title = "Shongo API", version = "v1"),
        security = @SecurityRequirement(name = "bearerAuth")
)
@SecurityScheme(
        name = "bearerAuth",
        type = SecuritySchemeType.HTTP,
        scheme = "bearer"
)
@ComponentScan(basePackages = {"org.springdoc"})
@Import({
        SpringDocConfiguration.class,
        SpringDocWebMvcConfiguration.class,
        SwaggerConfig.class,
        SwaggerUiConfigProperties.class,
        SwaggerUiOAuthProperties.class,
        JacksonAutoConfiguration.class
})
class OpenApiConfig implements WebMvcConfigurer {
    ...
}
\end{lstlisting}
Springdoc generates OpenAPI specification based on Spring web annotations by default, but Springdoc annotations such as \texttt{@Operation} or \texttt{@ApiResponses} can customize the resulting specification.
The resulting OpenAPI file is added to the REST server and is available at the \texttt{/v3/open-api} path \cite{springdoc}.

The Swagger UI (\Cref{sec:swagger}) is then used for user-friendly visualization of the OpenAPI specification mentioned above.
The user interface (web site) is acquired and saved in the project resources using the same \emph{springdoc-openapi} dependency as for OpenAPI specification. After that, the default OpenAPI path is set to \texttt{/v3/open-api}. Finally, these resources are added to the REST API server and made available at the \texttt{/swagger-ui/index.html} path.

\chapter{Final Api}
This chapter describes the final REST API implemented for the Shongo system.

\section{Users and Groups}
The client might need the information about users and groups in the Shongo system.
These resources are made available via endpoints implemented in \texttt{UserController}.

\subsection{Endpoints}
\begin{itemize}
    \item \textbf{\text{[GET]} /api/v1/users} -- Returns all users (\texttt{ListResponse<UserInformation>}) that match the \texttt{filter} query parameter and are part of group with \texttt{groupId} query parameter.
    \item \textbf{\text{[GET]} /api/v1/users/\{userId:.+\}} -- Returns information about a single user (\texttt{UserInformation}). The requested user is defined by the \texttt{userId} path variable.
    \item \textbf{\text{[GET]} /api/v1/groups} -- Returns all groups (\texttt{ListResponse<Group>}) that match the \texttt{filter} query parameter.
    \item \textbf{\text{[GET]} /api/v1/users/\{groupId:.+\}} -- Returns information about a single group (\texttt{Group}). The requested group is defined by the \texttt{groupId} path variable.
    \item \textbf{/api/v1/settings} -- Manages the user's settings. There are 2 available HTTP methods:
    \begin{description}
        \item \textbf{GET} -- Returns the user's predefined settings (\texttt{SettingsModel}). The user is defined by \texttt{SecurityToken} acquired from the \texttt{Authorization} HTTP header.
        \item \textbf{PUT} -- Updates the user's predefined settings to the settings (\texttt{SettingsModel}) obtained in request body. The user is defined by \texttt{SecurityToken} acquired from the \texttt{Authorization} HTTP header.
    \end{description}
\end{itemize}

\section{Resources}
Endpoints discussed in this section are endpoints concerning \texttt{Resource}s.
They are implemented in \texttt{ResourcesController} and available at \texttt{/api/v1/resources} path.

\subsection{List Roles}
\begin{table}[ht!]
    \begin{tabularx}{\textwidth}{llX}
        \toprule
        Name & Type & Description \\
        \midrule
        technology & QUERY & Filters resources by given \emph{TechnologyModel} \\  
        tag & QUERY & Filters resources by given \emph{tag}
        \end{tabularx}
    \caption{List resources parameters table.}
\end{table}
\begin{description}
    \item \textbf{HTTP request}\\
        \texttt{\text{[GET]} /api/v1/resources}
    \item \textbf{Description}\\
        Lists all available \texttt{Resource}s.
    \item \textbf{Response}\\
        \texttt{\text{[200 OK]} \text{[ResourceModel]}}
\end{description}
\subsection{Resource Capacity Utilization}
\begin{table}[ht!]
    \begin{tabularx}{\textwidth}{llX}
        \toprule
        Name & Type & Description \\
        \midrule
        interval\_from & QUERY & Computes interval from this date \\  
        interval\_to & QUERY & Computes interval to this date \\
        unit & QUERY & Divides interval into chunks as big as the \emph{unit} states \\
        refresh & QUERY & Forces refresh of utilization computation if \emph{true} \\
        start & QUERY & \emph{Start} the listing from this number \\  
        count & QUERY & \emph{Count} of items to return
        \end{tabularx}
    \caption{Resource utilization parameters table.}
\end{table}
\begin{description}
    \item \textbf{HTTP request}\\
        \texttt{\text{[GET]} /api/v1/resources/capacity\_utilizations}
    \item \textbf{Description}\\
        Lists \texttt{CapacityUtilization}s for all \texttt{Resources} in given interval, spited by \emph{unit} time period.
    \item \textbf{Response}\\
        \texttt{\text{[200 OK]} ListResponse<CapacityUtilization>}
\end{description}
\section{User Roles}
When a user decides to adjust the \texttt{UserRole}s of the reservation requests he has access to, he can use the endpoints implemented in \texttt{UserRoleController}.
The available endpoints include:
\begin{itemize}
    \item \textbf{/api/v1/reservation\_requests/\{id\}/roles} -- This endpoint manages user roles that are configured for the reservation request specified by the \texttt{id} path parameter. There are two available HTTP methods:
    \begin{description}
        \item \textbf{[GET]} -- Returns all roles (\texttt{ListResponse<UserRoleModel>}) that are configured for specified reservation request.
        \item \textbf{[POST]} -- Creates a new role defined by \texttt{UserRoleModel} acquired from the request body.
    \end{description}
    \item \textbf{[DELETE] /api/v1/reservation\_requests/\{id\}/roles/\{entityId\}} -- Deletes the configured user role specified by \texttt{entityId} path parameter from reservation request specified by \texttt{id} path parameter.
\end{itemize}

\section{Participants}
If a client wants to adjust the users or groups that can participate in reserved meetings, he could use the endpoints defined in  \texttt{Participant\-Controller}.
The available endpoints include:
\begin{itemize}
    \item \textbf{/api/v1/reservation\_requests/\{id:.+\}/participants} -- Manages participants configured for reservation request specified by \texttt{id} path parameter. There are two possible HTTP methods:
    \begin{description}
        \item \textbf{[GET]} -- Returns all participants (\texttt{ListResponse<Participant\-Model>}) that are configured for the reservation request specified by the \texttt{id} parameter.
        \item \textbf{[POST]} -- Creates a new participant defined by \texttt{ParticipantModel} acquired from the request body for the specified reservation request.
    \end{description}
    \item \textbf{/api/v1/reservation\_requests/\{id:.+\}/participants/\{participantId:.+\}} -- Manages the participant specified by \texttt{participantId} path parameter that is already configured for reservation request specified by \texttt{id} path parameter. There are two available HTTP methods:
    \begin{description}
        \item \textbf{[PUT]} -- Updates the existing participant's role with the participant role defined by the \texttt{role} (\texttt{ParticipantRole}) query parameter.
        \item \textbf{[DELETE]} -- Deletes the participant from the reservation request configuration.
    \end{description}
\end{itemize}

\section{Reservation Requests}
\subsection{List Reservation Requests}
\begin{table}[ht!]
    \begin{tabularx}{\textwidth}{llX}
        \toprule
        Name & Type & Description \\
        \midrule
        resource & QUERY & Filters entries that use given \emph{resource} \\
        type & QUERY & Filters entries with given \emph{type} \\
        search & QUERY & Filters entries that contains \emph{search} \\
        participant\_user\_id & QUERY &  Filters entries that have participant with id \emph{participant\_user\_id} \\
        user\_id & QUERY & Filters entries that owner's id equals \emph{user\_id} \\
        interval\_from & QUERY & Filters entries from this date \\  
        interval\_to & QUERY & Filters entries to this date \\
        technology & QUERY & Filters entries by given \emph{TechnologyModel} \\
        parentRequestId & QUERY & Filters entries that have \emph{parentRequestId} \\
        allocation\_state & QUERY & Filters entries that have \emph{allocation\_state} \\
        sort & QUERY & Sort entries by \emph{sort} parameter \\
        sort-desc & QUERY & Sorts entries in descending order if true \\
        start & QUERY & \emph{Start} the listing from this number \\  
        count & QUERY & \emph{Count} of records to return
        \end{tabularx}
    \caption{List roles parameters table.}
\end{table}
\begin{description}
    \item \textbf{HTTP request}\\
        \texttt{\text{[GET]} /api/v1/reservation\_requests}
    \item \textbf{Description}\\
        Lists \texttt{ReservationRequest}s.
    \item \textbf{Response}\\
        \texttt{\text{[200 OK]} ListResponse<ReservationRequest>}
\end{description}

\subsection{Create Reservation Requests}
\begin{description}
    \item \textbf{HTTP request}\\
        \texttt{\text{[POST]} /api/v1/reservation\_requests \texttt{ReservationRequest}}
    \item \textbf{Description}\\
        Creates \texttt{ReservationRequest} given in request body.
    \item \textbf{Response}\\
        \texttt{\text{[200 OK]}}
\end{description}

\subsection{Delete Reservation Requests}
\begin{description}
    \item \textbf{HTTP request}\\
        \texttt{\text{[DELETE]} /api/v1/reservation\_requests/\{id\} \texttt{ReservationRequest}}
    \item \textbf{Description}\\
        Creates \texttt{ReservationRequest} given in request body.
    \item \textbf{Response}\\
        \texttt{\text{[200 OK]}}
\end{description}

\section{Rooms}
The endpoints discussed in this section concern the reservation recordings and are defined in \texttt{RoomController}.
The available endpoints include:
\begin{itemize}
    \item \textbf{[GET] /api/v1/rooms} -- Returns all rooms (\texttt{ListResponse<Room\-Model>}) that can be filtered by the participation of a participant specified by the \texttt{participant-user-id} query parameter.
    \item \textbf{[GET] /api/v1/rooms/\{id:.+\}} -- Returns the detailed information about room (\texttt{RoomAuthorizedData}) that contains additional information about pins and aliases.
\end{itemize}

\section{Recordings}
Endpoints are defined in \texttt{RecordingController}.
\subsection{/api/v1/reservation\_requests/\{id\}/recordings}
Returns \texttt{ListResponse} of recordings from specific reservation request.
\begin{table}[ht!]
    % \centering
    \begin{tabularx}{\textwidth}{llX}
        \toprule
        Name & Type & Description \\
        \midrule
        id & PATH & Id of the reservation request \\ 
        start & QUERY & \emph{Start} the listing from this number \\  
        count & QUERY & \emph{Count} of records to return \\
        sort & QUERY & Sort entries by \emph{sort} parameter \\
        sort-desc & QUERY & Sorts entries in descending order if true \\
        \bottomrule
        \end{tabularx}
    \caption{Parameters table.}
\end{table}
% \begin{itemize}
%     \item \textbf{HTTP method} -- \texttt{GET}
%     \item \textbf{Parameters}
%         \begin{table}[h!]
%         % \centering
%         \begin{tabularx}{\textwidth}{llX}
%          \toprule
%          Name & Type & Description \\
%          \midrule
%          id & PATH & Id of the reservation request \\ 
%          start & QUERY & \emph{Start} the listing from this number \\  
%          count & QUERY & \emph{Count} of records to return \\
%          sort & QUERY & Sort entries by \emph{sort} parameter \\
%          sort-desc & QUERY & Sorts entries in descending order if true
%         \end{tabularx}
%         \caption{Parameters table.}
%         \end{table}
%     \item \textbf{Returns} -- 
% \end{itemize}
\subsection{/api/v1/reservation\_requests/\{id\}/recordings/\{recordingId\}}

\section{Report}
Endpoints concerning reporting of problem. Stored in \texttt{ReportController}.

\subsection{Report}
\begin{description}
    \item \textbf{HTTP request}\\
        \texttt{\text{[POST]} /api/v1/report \texttt{Report}}
    \item \textbf{Description}\\
        Sends \texttt{Report} to configured administrators.
    \item \textbf{Response}\\
        \texttt{\text{[200 OK]}}
\end{description}

\chapter{Conclusion}
In conclusion, new functional API was implemented for the Shongo system.
This API fulfills REST principles.
As a result, the new front-end \cite{drobnakm} is able to communicate with Shongo back-end, thus making the original \texttt{shongo-web-client} obsolete.
Moreover, the API makes the Shongo system available for any application or service outside of Shongo system.

The resulting REST API is documented with \texttt{Javadoc} and \hyperref[sec:openapi]{OpenAPI specification} is generated and made available on REST server.
In addition, this specification is added to \hyperref[sec:spring]{Spring UI}, which is also deployed on REST server.

The end.

\appendix %% Start the appendices.
\chapter{Attached Files} \label{apx:files}
This work consists of the electronic version of the thesis and a zip archive \texttt{shongo.zip}, both publicly accessible in the Information System of Masaryk University\footnote{\url{https://is.muni.cz/th/jhvt4/}}.
The archive consists of 2 parts:
\begin{itemize}
    \item The source code of the Shongo system with implemented REST API is available the in \texttt{shongo} directory.
    \item Final generated OpenAPI specification (see \Cref{sec:openapi}) for implemented REST API is available in the \texttt{openapi.json} file. Also, pre-configured Swagger UI (see \Cref{sec:swagger}) is attached, for pleasant viewing of the specification.
\end{itemize}


\end{document}
