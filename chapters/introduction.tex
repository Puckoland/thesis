%% The \chapter* command can be used to produce unnumbered chapters:
\chapter*{Introduction}
%% Unlike \chapter, \chapter* does not update the headings and does not
%% enter the chapter to the table of contents. I we want correct
%% headings and a table of contents entry, we must add them manually:
\markright{\textsc{Introduction}}
\addcontentsline{toc}{chapter}{Introduction}

Videoconferences have become a big topic in the last few years. With the globalization of the companies and an increasing number of people working remotely, the demand for methods that can be used for remote communication is growing in direct proportion.
Especially now, when the world pandemic stormed the lives of all humankind and limited the physical contact between people, they can appreciate the impact of this technology. The children and students can learn from their homes by communicating with their teachers \enquote{live} in the videoconferences as mentioned earlier. Jobs that require communication considerably but do not require human contact can take full advantage of this technology.
However, the technology is bound to the available resources used for the videoconferences.

The academic community, as well as any large company, has a certain amount of resources at its disposal. For effective usage, the management of these resources is required.
For this purpose, the CESNET association created the Shongo system. The Shongo system manages available resources, makes reservations for these resources, and notifies participants about an upcoming videoconference.

This thesis was assigned to implement REST API for the Controller module (see \Cref{controller}) of the Shongo system. The rest of the introduction focuses on describing this work's chapters.

In \Cref{cha:shongo}, the current reservation system Shongo will be introduced. Furthermore, this chapter discusses the current state of the Shongo architecture, its flaws, and how this thesis will change the architecture after the work on this thesis is done.

Next, the technologies used to finalize the assignment of this thesis have to be adequately explained. This explanation and a brief introduction to RESTful API can be found in \Cref{cha:technologies}.

When the reader is acquainted with the technologies used, the thesis continues to explain how they were used to complete the assignment in \Cref{cha:implementation}. This chapter describes the structure of the code added to the Shongo system --- the configuration, implementation and documentation of the REST API as the assignment requests.

Last but not least, the final API is described in \Cref{cha:api}, so the reader can imagine how the client can use the implemented REST API to use the Shongo system described in \Cref{cha:shongo}.

Finally, \Cref{cha:conclusion} summarizes the thesis and presents the thoughts about what can be done next.
