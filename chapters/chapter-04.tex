\chapter{Mathematical equations}
\label{chap:matheq}
\TeX{} comes pre-packed with the ability to typeset inline
equations, such as $\mathrm{e}^{ix}=\cos x+i\sin x$, and display
equations, such as \[
  \mathbf{A}^{-1} = \begin{bmatrix}
  a & b \\ c & d \\
  \end{bmatrix}^{-1} =
  \frac{1}{\det(\mathbf{A})} \begin{bmatrix}
  \,\,\,d & \!\!-b \\ -c & \,a \\
  \end{bmatrix} =
  \frac{1}{ad - bc} \begin{bmatrix}
  \,\,\,d & \!\!-b \\ -c & \,a \\
  \end{bmatrix}.
\] \LaTeX{} defines the automatically numbered \texttt{equation}
environment:
\begin{equation}
  \gamma Px = PAx = PAP^{-1}Px.
\end{equation}
The package \textsf{amsmath} provides several additional
environments that can be used to typeset complex equations:
\begin{enumerate}
  \item An equation can be spread over multiple lines using the
    \texttt{multline} environment:
    \begin{multline}
      a + b + c + d + e + f + b + c + d + e + f + b + c + d + e +
f \\
      + f + g + h + i + j + k + l + m + n + o + p + q
    \end{multline}

  \item Several aligned equations can be typeset using the
    \texttt{align} environment:
    \begin{align}
              a + b &= c + d     \\
                  u &= v + w + x \\[1ex]
      i + j + k + l &= m
    \end{align}

  \item The \texttt{alignat} environment is similar to
    \texttt{align}, but it doesn't insert horizontal spaces between
    the individual columns:
    \begin{alignat}{2}
      a + b + c &+ d       &   &= 0 \\
              e &+ f + g   &   &= 5
    \end{alignat}

  \item Much like chapter, sections, tables, figures, or list
    items, equations -- such as \eqref{eq:first} and
    \eqref{eq:mine} -- can also be labeled and referenced:
    \begin{alignat}{4}
      b_{11}x_1 &+ b_{12}x_2  &  &+ b_{13}x_3  &  &             &
        &= y_1,                   \label{eq:first} \\
      b_{21}x_1 &+ b_{22}x_2  &  &             &  &+ b_{24}x_4  &
        &= y_2. \tag{My equation} \label{eq:mine}
    \end{alignat}

  \item The \texttt{gather} environment makes it possible to
    typeset several equations without any alignment:
    \begin{gather}
      \psi = \psi\psi, \\
      \eta = \eta\eta\eta\eta\eta\eta, \\
      \theta = \theta.
    \end{gather}

  \item Several cases can be typeset using the \texttt{cases}
    environment:
    \begin{equation}
      |y| = \begin{cases}
        \phantom-y & \text{if }z\geq0, \\
                -y & \text{otherwise}.
      \end{cases}
    \end{equation}
\end{enumerate}
For the complete list of environments and commands, consult the
\textsf{amsmath} package manual\footnote{
  See \url{http://mirrors.ctan.org/macros/latex/required/amsmath/amsldoc.pdf}.
  The \texttt{\textbackslash url} command is provided by the
  package \textsf{url}.
}.
