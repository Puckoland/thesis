\chapter{Inserting the bibliography}
After linking a bibliography data\-base files to the document using
the \verb"\"\texttt{thesis\discretionary{-}{}{}setup\{bib\discretionary{=}{=}{=}%
\{\textit{file1},\textit{file2},\,\ldots\,\}\}} command, you can
start citing the entries. This is just dummy text
\parencite{borgman03} lightly sprinkled with citations
\parencite[p.~123]{greenberg98}. Several sources can be cited at
once: \cite{borgman03,greenberg98,thanh01}.
\citetitle{greenberg98} was written by \citeauthor{greenberg98} in
\citeyear{greenberg98}. We can also produce \textcite{greenberg98}%
\ or %% Let us define a compound command:
\def\citeauthoryear#1{(\textcite{#1},~\citeyear{#1})}%
\citeauthoryear{greenberg98}%
. The full bibliographic citation is:
\emph{\fullcite{greenberg98}}. We can easily insert a bibliographic
citation into the footnote\footfullcite{greenberg98}.

The \verb"\nocite" command will not generate any
output\nocite{muni}, but it will insert its arguments into
the bibliography. The \verb"\nocite{*}" command will insert all the
records in the bibliography database file into the bibliography.
Try uncommenting the command
%% \nocite{*}
and watch the bibliography section come apart at the seams.

When typesetting the document for the first time, citing a
\texttt{work} will expand to [\textbf{work}] and the
\verb"\printbibliography" command will produce no output. It is now
necessary to generate the bibliography by running \texttt{biber
\jobname.bcf} from the command line and then by typesetting the
document again twice. During the first run, the bibliography
section and the citations will be typeset, and in the second run,
the bibliography section will appear in the table of contents.

The \texttt{biber} command needs to be executed from within the
directory, where the \LaTeX\ source file is located. In Windows,
the command line can be opened in a directory by holding down the
\textsf{Shift} key and by clicking the right mouse button while
hovering the cursor over a directory.  Select the \textsf{Open
Command Window Here} option in the context menu that opens shortly
afterwards.

With online services -- such as Overleaf -- or when using an
automatic tool -- such as \LaTeX MK -- all commands are executed
automatically. When you omit the \verb"\printbibliography" command,
its location will be decided by the template.
