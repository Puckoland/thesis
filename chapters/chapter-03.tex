\chapter{Floats and references}
\begin{figure}
  \begin{center}
    %% PNG and JPG images can be inserted into the document as well,
    %% but their resolution needs to be adequate. The minimum is
    %% about 100 pixels per 1 centimeter or 300 pixels per 1 inch.
    %% That means that a JPG or PNG image typeset at 4 × 4 cm should
    %% be 400 × 400 px large at the bare minimum.
    %%
    %% The optimum is about 250 pixels per 1 centimeter or 600
    %% pixels per 1 inch. That means that a JPG or PNG image typeset
    %% at 4 × 4 cm should be 1000 × 1000 px large or larger.
    \includegraphics[width=6.3cm]{fithesis/logo/mu/fithesis-base-english-color}
  \end{center}
  \caption{The logo of \acrlong{MU} at 6.3\,cm}
  \label{fig:mulogo1}
\end{figure}

\begin{figure}
  \begin{center}
    \begin{minipage}{.5\textwidth}
      \includegraphics[width=\textwidth]{fithesis/logo/mu/fithesis-base-english-color}
    \end{minipage}
    \hfill  % Fill the horizontal space between the images
    \begin{minipage}{.33\textwidth}
      \includegraphics[width=\textwidth]{fithesis/logo/mu/fithesis-base-english-color} \\[1em]
      \includegraphics[width=\textwidth]{fithesis/logo/mu/fithesis-base-english-color}
    \end{minipage}
  \end{center}
\caption{The logo of \acrlong{MU} at $\frac12$ and
    $\frac13$ of text width}
  \label{fig:mulogo2}
\end{figure}

\begin{table}
  \begin{tabularx}{\textwidth}{lllX}
    \toprule
    Day & Min Temp & Max Temp & Summary \\
    \midrule
    Monday & $13^{\circ}\mathrm{C}$ & $21^\circ\mathrm{C}$ & A
    clear day with low wind and no adverse current advisories. \\
    Tuesday & $11^{\circ}\mathrm{C}$ & $17^\circ\mathrm{C}$ & A
    trough of low pressure will come from the northwest. \\
    Wednesday & $10^{\circ}\mathrm{C}$ &
    $21^\circ\mathrm{C}$ & Rain will spread to all parts during the
    morning. \\
    \bottomrule
  \end{tabularx}
  \caption{A weather forecast}
  \label{tab:weather}
\end{table}

The logo of \gls{MU} is shown in Figure \ref{fig:mulogo1} and
Figure \ref{fig:mulogo2} at pages \pageref{fig:mulogo1} and
\pageref{fig:mulogo2}. The weather forecast is shown in Table
\ref{tab:weather} at page \pageref{tab:weather}. The following
chapter is Chapter \ref{chap:matheq} and starts at page
\pageref{chap:matheq}.  Items \ref{item:star1}, \ref{item:star2},
and \ref{item:star3} are starred in the following list:
\begin{compactenum}
  \item some text
  \item some other text
  \item $\star$ \label{item:star1}
  \begin{compactenum}
    \item some text
    \item $\star$ \label{item:star2}
    \item some other text
    \begin{compactenum}
      \item some text
      \item some other text
      \item yet another piece of text
      \item $\star$ \label{item:star3}
    \end{compactenum}
    \item yet another piece of text
  \end{compactenum}
  \item yet another piece of text
\end{compactenum}
If your reference points to a place that has not yet been typeset,
the \verb"\ref" command will expand to \textbf{??} during the first
run of
\texttt{pdflatex \jobname.tex}
and a second run is going to be needed for the references to
resolve. With online services -- such as \Gls{Overleaf} -- this is
performed automatically.
