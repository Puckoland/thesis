\section{What to Change and Why}

The main reason for changes is that the architecture concentrates on modularity; however, the system cannot be observed or managed from external sources.
The second reason is the primary user interface --- the \hyperref[webclient]{\emph{Web Client}}. Web MVC and server-side rendering from current Web Client state are not as powerful as today's JavaScript frameworks like Angular or React.
That is why developing a new front-end became a goal of another thesis. \cite{drobnakm}
Nevertheless, until now, the whole communication took place within the system via Spring Framework in Java. In order to achieve communication between Shongo and the new Web Client (and possibly other systems or services), which is not part of the Shongo system, a new API has to be implemented.

\scalegraphics{assets/shongo_architecture_new}{New Shongo architecture}{architecture_new}

Comparing the old system architecture shown in \Cref{fig:architecture} and the new architecture shown in \Cref{fig:architecture_new} shows that the \emph{Web Client} component was moved out of the Shongo system. The new \emph{Web Client} is now developed and maintained separately at the \href{https://github.com/shongo/shongo-frontend}{new Github repository}. Another thesis was devoted to this part of the changes \cite{drobnakm}.
Additionally, the \emph{Web Client} communicates with the \hyperref[controller]{\emph{Controller}} via \emph{HTTP} requests for the \emph{REST API} server sub-component instead of previous direct communication via \emph{XML-RPC} services. The implementation of this \emph{REST API} is the subject of this thesis.
