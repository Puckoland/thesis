\section{Resources}
Endpoints discussed in this section are endpoints concerning \texttt{Resource}s.
They are implemented in \texttt{ResourcesController} and available at \texttt{/api/v1/resources} path.

\subsection{List Roles}
\begin{table}[ht!]
    \begin{tabularx}{\textwidth}{llX}
        \toprule
        Name & Type & Description \\
        \midrule
        technology & QUERY & Filters resources by given \emph{TechnologyModel} \\  
        tag & QUERY & Filters resources by given \emph{tag}
        \end{tabularx}
    \caption{List resources parameters table.}
\end{table}
\begin{description}
    \item \textbf{HTTP request}\\
        \texttt{\text{[GET]} /api/v1/resources}
    \item \textbf{Description}\\
        Lists all available \texttt{Resource}s.
    \item \textbf{Response}\\
        \texttt{\text{[200 OK]} \text{[ResourceModel]}}
\end{description}
\subsection{Resource Capacity Utilization}
\begin{table}[ht!]
    \begin{tabularx}{\textwidth}{llX}
        \toprule
        Name & Type & Description \\
        \midrule
        interval\_from & QUERY & Computes interval from this date \\  
        interval\_to & QUERY & Computes interval to this date \\
        unit & QUERY & Divides interval into chunks as big as the \emph{unit} states \\
        refresh & QUERY & Forces refresh of utilization computation if \emph{true} \\
        start & QUERY & \emph{Start} the listing from this number \\  
        count & QUERY & \emph{Count} of items to return
        \end{tabularx}
    \caption{Resource utilization parameters table.}
\end{table}
\begin{description}
    \item \textbf{HTTP request}\\
        \texttt{\text{[GET]} /api/v1/resources/capacity\_utilizations}
    \item \textbf{Description}\\
        Lists \texttt{CapacityUtilization}s for all \texttt{Resources} in given interval, spited by \emph{unit} time period.
    \item \textbf{Response}\\
        \texttt{\text{[200 OK]} ListResponse<CapacityUtilization>}
\end{description}