\section{Users and Groups}
The client might need information about users and groups in the Shongo system.
These resources are made available via endpoints implemented in \texttt{UserController}.
The available endpoints include:
\begin{itemize}
    \item \textbf{\text{[GET]} /api/v1/users} -- Returns all users (\texttt{ListResponse<UserInformation>}) that match the \texttt{filter} query parameter and are part of the group with the \texttt{groupId} query parameter.
    \item \textbf{\text{[GET]} /api/v1/users/\{userId:.+\}} -- Returns information about a single user (\texttt{UserInformation}). The \texttt{userId} path variable defines the requested user.
    \item \textbf{\text{[GET]} /api/v1/groups} -- Returns all groups (\texttt{ListResponse<Group>}) that match the \texttt{filter} query parameter.
    \item \textbf{\text{[GET]} /api/v1/users/\{groupId:.+\}} -- Returns information about a single group (\texttt{Group}). The \texttt{groupId} path variable defines the requested group.
    \item \textbf{/api/v1/settings} -- Manages the user’s settings. \texttt{SecurityToken} acquired from the \texttt{Authorization} HTTP header defines the user whose settings are being addressed. There are two available HTTP methods:
    \begin{description}
        \item \textbf{GET} -- Returns the user’s predefined settings (\texttt{SettingsModel}).
        \item \textbf{PUT} -- Updates the user's predefined settings to the settings (\texttt{SettingsModel}) received in the request body.
    \end{description}
\end{itemize}
