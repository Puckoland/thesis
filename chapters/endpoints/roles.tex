\section{User Roles}
Endpoints discussed in this section are endpoints concerning \texttt{UserRole}.
They are implemented in \texttt{UserRoleController}.
\subsection{List Roles}
\begin{table}[ht!]
    % \centering
    \begin{tabularx}{\textwidth}{llX}
        \toprule
        Name & Type & Description \\
        \midrule
        id & PATH & Id of the reservation request \\ 
        start & QUERY & \emph{Start} the listing from this number \\  
        count & QUERY & \emph{Count} of records to return
        \end{tabularx}
    \caption{List roles parameters table.}
\end{table}
\begin{description}
    \item \textbf{HTTP request}\\
        \texttt{\text{[GET]} /api/v1/reservation\_requests/\{id\}/roles}
    \item \textbf{Description}\\
        Lists \texttt{UserRole}s for reservation request by \texttt{id}.
    \item \textbf{Response}\\
        \texttt{\text{[200 OK]} ListResponse<UserRole>}
\end{description}
\subsection{Add Role}
\begin{table}[ht!]
    % \centering
    \begin{tabularx}{\textwidth}{llX}
        \toprule
        Name & Type & Description \\
        \midrule
        id & PATH & Id of the reservation request \\ 
        \end{tabularx}
    \caption{Add role parameters table.}
\end{table}
\begin{description}
    \item \textbf{Description}\\
        Creates a new \texttt{UserRole} by request body for reservation request by \texttt{id}.
    \item \textbf{HTTP request}\\
        \texttt{\text{[POST]} /api/v1/reservation\_requests/\{id\}/roles}\\
        \texttt{UserRole}
    \item \textbf{Response}\\
        \texttt{\text{[200 OK]}}
\end{description}
\subsection{Delete Role}
\begin{table}[ht!]
    % \centering
    \begin{tabularx}{\textwidth}{llX}
        \toprule
        Name & Type & Description \\
        \midrule
        id & PATH & Id of the reservation request \\
        entityId & PATH & Id of the user/group of role \\
        \end{tabularx}
    \caption{Delete role parameters table.}
\end{table}
\begin{description}
    \item \textbf{Description}\\
        Deletes the \texttt{UserRole} by \texttt{entityId} from reservation request by \texttt{id}.
    \item \textbf{HTTP request}\\
        \texttt{\text{[DELETE]} /api/v1/reservation\_requests/\{id\}/roles/\{entityId\}}
    \item \textbf{Response}\\
        \texttt{\text{[200 OK]}}
\end{description}