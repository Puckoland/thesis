\section{Recordings}
The endpoints discussed in this section concern the reservation recordings. They are defined in \texttt{RecordingController}.

\subsection{List Recordings}
\begin{description}
    \item \textbf{HTTP request}\\
        \texttt{\text{[GET]} /api/v1/reservation\_requests/\{id\}/recordings}
    \item \textbf{Description}\\
        Returns \texttt{ListResponse} of recordings from reservation request specified by \texttt{id} parameter.
    \item \textbf{Response}\\
        \texttt{\text{[200 OK]} ListResponse<RecordingModel>}
\end{description}
\begin{table}[ht!]
    % \centering
    \begin{tabularx}{\textwidth}{llX}
        \toprule
        Name & Type & Description \\
        \midrule
        id & PATH & Id of the reservation request \\ 
        start & QUERY & \emph{Start} the listing from this number \\  
        count & QUERY & \emph{Count} of records to return \\
        sort & QUERY & Sort entries by \emph{sort} parameter \\
        sort-desc & QUERY & Sorts entries in descending order if true \\
        \bottomrule
        \end{tabularx}
    \caption{Parameters table.}
\end{table}
\subsection{Delete Recording}
\begin{description}
    \item \textbf{HTTP request}\\
        \texttt{\text{[DELETE]} /api/v1/reservation\_requests/\{id\}\\/recordings/\{recordingId\}}
    \item \textbf{Description}\\
        Deletes the recording with \texttt{recordingId} from reservation request specified by \texttt{id} parameter.
    \item \textbf{Response}\\
        \texttt{\text{[200 OK]}}
\end{description}
\begin{table}[ht!]
    % \centering
    \begin{tabularx}{\textwidth}{llX}
        \toprule
        Name & Type & Description \\
        \midrule
        id & PATH & Id of the reservation request \\ 
        recordingId & PATH & Id of the recording to delete \\
        \bottomrule
        \end{tabularx}
    \caption{Parameters table.}
\end{table}
