\section{Participants}
The endpoints discussed in this section concern the participants. They are defined in \texttt{ParticipantController}.

\subsection{List Participants}
\begin{description}
    \item \textbf{HTTP request}\\
        \texttt{\text{[GET]} /api/v1/reservation\_requests/\{id:.+\}/participants}
    \item \textbf{Description}\\
        Returns \texttt{ListResponse} of participants from reservation request specified by \texttt{id} parameter.
    \item \textbf{Response}\\
        \texttt{\text{[200 OK]} ListResponse<ParticipantModel>}
\end{description}
\begin{table}[ht!]
    \begin{tabularx}{\textwidth}{llX}
        \toprule
        Name & Type & Description \\
        \midrule
        id & PATH & id of reservation request \\
        start & QUERY & \emph{Start} the listing from this number \\  
        count & QUERY & \emph{Count} of records to return \\
        \bottomrule
        \end{tabularx}
    \caption{List participants parameters table.}
\end{table}

\subsection{Create Participant}
\begin{description}
    \item \textbf{HTTP request}\\
        \texttt{\text{[POST]} /api/v1/reservation\_requests/\{id:.+\}/participants}
    \item \textbf{Description}\\
        Adds \texttt{ParticipantModel} from request body to reservation request specified by \texttt{id} parameter.
    \item \textbf{Response}\\
        \texttt{\text{[200 OK]}}
\end{description}
\begin{table}[ht!]
    \begin{tabularx}{\textwidth}{llX}
        \toprule
        Name & Type & Description \\
        \midrule
        id & PATH & id of reservation request \\
        \bottomrule
        \end{tabularx}
    \caption{Create participant parameters table.}
\end{table}

\subsection{Edit Participant}
\begin{description}
    \item \textbf{HTTP request}\\
        \texttt{\text{[PUT]} /api/v1/reservation\_requests/\{id:.+\}\\
        /participants/{participantId:.+} ParticipantModel}
    \item \textbf{Description}\\
        Replaces \texttt{ParticipantModel} with \texttt{participantId} by \texttt{ParticipantModel} from request body.
    \item \textbf{Response}\\
        \texttt{\text{[200 OK]}}
\end{description}
\begin{table}[ht!]
    \begin{tabularx}{\textwidth}{llX}
        \toprule
        Name & Type & Description \\
        \midrule
        id & PATH & id of reservation request \\
        participantId & PATH & id of participant to edit \\
        \bottomrule
        \end{tabularx}
    \caption{Edit participant parameters table.}
\end{table}

\subsection{Delete Participant}
\begin{description}
    \item \textbf{HTTP request}\\
        \texttt{\text{[DELETE]} /api/v1/reservation\_requests/\{id:.+\}/participants/{participantId:.+}}
    \item \textbf{Description}\\
        Deletes \texttt{ParticipantModel} with \texttt{participantId} from reservation request specified by \texttt{id} parameter.
    \item \textbf{Response}\\
        \texttt{\text{[200 OK]}}
\end{description}
\begin{table}[ht!]
    \begin{tabularx}{\textwidth}{llX}
        \toprule
        Name & Type & Description \\
        \midrule
        id & PATH & id of reservation request \\
        participantId & PATH & id of participant to delete \\
        \bottomrule
        \end{tabularx}
    \caption{Delete participant parameters table.}
\end{table}
