\section{Participants}
If a client wants to adjust the users or groups that can participate in reserved meetings, he can use the endpoints defined in  \texttt{Participant\-Controller}.
The available endpoints include:
\begin{itemize}
    \item \textbf{/api/v1/reservation\_requests/\{id:.+\}/participants} -- Manages participants configured for reservation request specified by \texttt{id} path parameter. There are two possible HTTP methods:
    \begin{description}
        \item \textbf{[GET]} -- Returns all participants (\texttt{ListResponse<Participant\-Model>}) that are configured for the reservation request specified by the \texttt{id} parameter.
        \item \textbf{[POST]} -- Creates a new participant defined by \texttt{ParticipantModel} acquired from the request body for the specified reservation request.
    \end{description}
    \item \textbf{/api/v1/reservation\_requests/\{id:.+\}/participants/\{participantId:.+\}} -- Manages the participant specified by \texttt{participantId} path parameter that is already configured for reservation request specified by \texttt{id} path parameter. There are two available HTTP methods:
    \begin{description}
        \item \textbf{[PUT]} -- Updates the existing participant's role with the participant role defined by the \texttt{role} (\texttt{ParticipantRole}) query parameter.
        \item \textbf{[DELETE]} -- Deletes the participant from the reservation request configuration.
    \end{description}
\end{itemize}
