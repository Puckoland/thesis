\section{Users and Groups}
The client might need the information about users and groups in the Shongo system.
These resources are made available via endpoints implemented in \texttt{UserController}.

\subsection{Endpoints}
\begin{itemize}
    \item \textbf{/api/v1/users} -- Returns all users that match the \texttt{filter} and query parameter and are part of \texttt{groupId} query parameter.
    \item \textbf{/api/v1/users/\{userId:.+\}} -- Returns information about a single user. The requested user is defined by the \texttt{userId} path variable.
    \item \textbf{/api/v1/groups} -- Returns all groups that match the \texttt{filter} query parameter.
    \item \textbf{/api/v1/users/\{groupId:.+\}} -- Returns information about a single group. The requested group is defined by the \texttt{groupId} path variable.
    \item \textbf{/api/v1/settings} -- Manages the user's settings. There are 2 available HTTP methods:
    \begin{description}
        \item \textbf{GET} -- Returns the user's predefined settings. The user is acquired by \texttt{SecurityToken} from the \texttt{Authorization} HTTP header.
        \item \textbf{PUT} -- Updates the user's predefined settings to the settings obtained in request body. The user is acquired by \texttt{SecurityToken} from the \texttt{Authorization} HTTP header.
    \end{description}
\end{itemize}


\section{Recordings}
Endpoints are defined in \texttt{RecordingController}.
\subsection{/api/v1/reservation\_requests/\{id\}/recordings}
Returns \texttt{ListResponse} of recordings from specific reservation request.
\begin{table}[ht!]
    % \centering
    \begin{tabularx}{\textwidth}{llX}
        \toprule
        Name & Type & Description \\
        \midrule
        id & PATH & Id of the reservation request \\ 
        start & QUERY & \emph{Start} the listing from this number \\  
        count & QUERY & \emph{Count} of records to return \\
        sort & QUERY & Sort entries by \emph{sort} parameter \\
        sort-desc & QUERY & Sorts entries in descending order if true \\
        \bottomrule
        \end{tabularx}
    \caption{Parameters table.}
\end{table}
% \begin{itemize}
%     \item \textbf{HTTP method} -- \texttt{GET}
%     \item \textbf{Parameters}
%         \begin{table}[h!]
%         % \centering
%         \begin{tabularx}{\textwidth}{llX}
%          \toprule
%          Name & Type & Description \\
%          \midrule
%          id & PATH & Id of the reservation request \\ 
%          start & QUERY & \emph{Start} the listing from this number \\  
%          count & QUERY & \emph{Count} of records to return \\
%          sort & QUERY & Sort entries by \emph{sort} parameter \\
%          sort-desc & QUERY & Sorts entries in descending order if true
%         \end{tabularx}
%         \caption{Parameters table.}
%         \end{table}
%     \item \textbf{Returns} -- 
% \end{itemize}
\subsection{/api/v1/reservation\_requests/\{id\}/recordings/\{recordingId\}}


\section{Report}
Endpoints concerning reporting of problem. Stored in \texttt{ReportController}.

\subsection{Report}
\begin{description}
    \item \textbf{HTTP request}\\
        \texttt{\text{[POST]} /api/v1/report \texttt{Report}}
    \item \textbf{Description}\\
        Sends \texttt{Report} to configured administrators.
    \item \textbf{Response}\\
        \texttt{\text{[200 OK]}}
\end{description}


\section{Reservation Requests}
\subsection{List Reservation Requests}
\begin{table}[ht!]
    \begin{tabularx}{\textwidth}{llX}
        \toprule
        Name & Type & Description \\
        \midrule
        resource & QUERY & Filters entries that use given \emph{resource} \\
        type & QUERY & Filters entries with given \emph{type} \\
        search & QUERY & Filters entries that contains \emph{search} \\
        participant\_user\_id & QUERY &  Filters entries that have participant with id \emph{participant\_user\_id} \\
        user\_id & QUERY & Filters entries that owner's id equals \emph{user\_id} \\
        interval\_from & QUERY & Filters entries from this date \\  
        interval\_to & QUERY & Filters entries to this date \\
        technology & QUERY & Filters entries by given \emph{TechnologyModel} \\
        parentRequestId & QUERY & Filters entries that have \emph{parentRequestId} \\
        allocation\_state & QUERY & Filters entries that have \emph{allocation\_state} \\
        sort & QUERY & Sort entries by \emph{sort} parameter \\
        sort-desc & QUERY & Sorts entries in descending order if true \\
        start & QUERY & \emph{Start} the listing from this number \\  
        count & QUERY & \emph{Count} of records to return
        \end{tabularx}
    \caption{List roles parameters table.}
\end{table}
\begin{description}
    \item \textbf{HTTP request}\\
        \texttt{\text{[GET]} /api/v1/reservation\_requests}
    \item \textbf{Description}\\
        Lists \texttt{ReservationRequest}s.
    \item \textbf{Response}\\
        \texttt{\text{[200 OK]} ListResponse<ReservationRequest>}
\end{description}

\subsection{Create Reservation Requests}
\begin{description}
    \item \textbf{HTTP request}\\
        \texttt{\text{[POST]} /api/v1/reservation\_requests \texttt{ReservationRequest}}
    \item \textbf{Description}\\
        Creates \texttt{ReservationRequest} given in request body.
    \item \textbf{Response}\\
        \texttt{\text{[200 OK]}}
\end{description}

\subsection{Delete Reservation Requests}
\begin{description}
    \item \textbf{HTTP request}\\
        \texttt{\text{[DELETE]} /api/v1/reservation\_requests/\{id\} \texttt{ReservationRequest}}
    \item \textbf{Description}\\
        Creates \texttt{ReservationRequest} given in request body.
    \item \textbf{Response}\\
        \texttt{\text{[200 OK]}}
\end{description}


\section{Resources}
Endpoints discussed in this section are endpoints concerning \texttt{Resource}.
They are implemented in \texttt{ResourcesController} and available at \texttt{/api/v1/resources} path.
\subsection{List Roles}
\begin{table}[ht!]
    \begin{tabularx}{\textwidth}{llX}
        \toprule
        Name & Type & Description \\
        \midrule
        technology & QUERY & Filters resources by given \emph{TechnologyModel} \\  
        tag & QUERY & Filters resources by given \emph{tag}
        \end{tabularx}
    \caption{List resources parameters table.}
\end{table}
\begin{description}
    \item \textbf{HTTP request}\\
        \texttt{\text{[GET]} /api/v1/resources}
    \item \textbf{Description}\\
        Lists all available \texttt{Resource}s.
    \item \textbf{Response}\\
        \texttt{\text{[200 OK]} \text{[Resource]}}
\end{description}
\subsection{Resource Capacity Utilization}
\begin{table}[ht!]
    \begin{tabularx}{\textwidth}{llX}
        \toprule
        Name & Type & Description \\
        \midrule
        interval\_from & QUERY & Computes interval from this date \\  
        interval\_to & QUERY & Computes interval to this date \\
        unit & QUERY & Divides interval into chunks as big as the \emph{unit} states \\
        refresh & QUERY & Forces refresh of utilization computation if \emph{true} \\
        start & QUERY & \emph{Start} the listing from this number \\  
        count & QUERY & \emph{Count} of items to return
        \end{tabularx}
    \caption{Resource utilization parameters table.}
\end{table}
\begin{description}
    \item \textbf{HTTP request}\\
        \texttt{\text{[GET]} /api/v1/resources/capacity\_utilizations}
    \item \textbf{Description}\\
        Lists \texttt{CapacityUtilization}s for all \texttt{Resources} in given interval, spited by \emph{unit}.
    \item \textbf{Response}\\
        \texttt{\text{[200 OK]} ListResponse<CapacityUtilization>}
\end{description}


\section{User Roles}
Endpoints discussed in this section are endpoints concerning \texttt{UserRole}.
They are implemented in \texttt{UserRoleController}.
\subsection{List Roles}
\begin{table}[ht!]
    % \centering
    \begin{tabularx}{\textwidth}{llX}
        \toprule
        Name & Type & Description \\
        \midrule
        id & PATH & Id of the reservation request \\ 
        start & QUERY & \emph{Start} the listing from this number \\  
        count & QUERY & \emph{Count} of records to return
        \end{tabularx}
    \caption{List roles parameters table.}
\end{table}
\begin{description}
    \item \textbf{HTTP request}\\
        \texttt{\text{[GET]} /api/v1/reservation\_requests/\{id\}/roles}
    \item \textbf{Description}\\
        Lists \texttt{UserRole}s for reservation request by \texttt{id}.
    \item \textbf{Response}\\
        \texttt{\text{[200 OK]} ListResponse<UserRole>}
\end{description}
\subsection{Add Role}
\begin{table}[ht!]
    % \centering
    \begin{tabularx}{\textwidth}{llX}
        \toprule
        Name & Type & Description \\
        \midrule
        id & PATH & Id of the reservation request \\ 
        \end{tabularx}
    \caption{Add role parameters table.}
\end{table}
\begin{description}
    \item \textbf{HTTP request}\\
        \texttt{\text{[POST]} /api/v1/reservation\_requests/\{id\}/roles}\\
        \texttt{UserRole}
    \item \textbf{Description}\\
        Creates a new \texttt{UserRole} by request body for reservation request by \texttt{id}.
    \item \textbf{Response}\\
        \texttt{\text{[200 OK]}}
\end{description}
\subsection{Delete Role}
\begin{table}[ht!]
    % \centering
    \begin{tabularx}{\textwidth}{llX}
        \toprule
        Name & Type & Description \\
        \midrule
        id & PATH & Id of the reservation request \\
        entityId & PATH & Id of the user/group of role \\
        \end{tabularx}
    \caption{Delete role parameters table.}
\end{table}
\begin{description}
    \item \textbf{HTTP request}\\
        \texttt{\text{[DELETE]} /api/v1/reservation\_requests/\{id\}/roles/\{entityId\}}
    \item \textbf{Description}\\
        Deletes the \texttt{UserRole} by \texttt{entityId} from reservation request by \texttt{id}.
    \item \textbf{Response}\\
        \texttt{\text{[200 OK]}}
\end{description}

