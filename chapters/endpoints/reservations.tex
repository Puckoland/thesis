\section{Reservation Requests}
The following endpoints are the most important since they manage the central part of the Shongo system --- reservation requests.
The available endpoints include:
\begin{itemize}
    \item \textbf{/api/v1/reservation\_requests} -- Manages the reservation requests. There are two available HTTP methods:
    \begin{description}
        \item \textbf{[GET]} -- Returns all reservation requests (\texttt{ListResponse<Reser\-vationRequestModel>}). However, they can be filtered by many query parameters closely described in \Cref{tab:res}.
        \item \textbf{[POST]} -- Creates a new reservation request defined by \texttt{Reser\-vationRequestCreateModel} acquired from the request body.
    \end{description}
    \item \textbf{/api/v1/reservation\_requests/\{id:.+\}} -- Manages the reservation request specified by \texttt{id} path parameter. There are two available HTTP methods:
    \begin{description}
        \item \textbf{[GET]} -- Returns detailed information about the reservation request (\texttt{ReservationRequestDetailModel}). It contains additional information about the current state of the reservation request, its authorized data (\texttt{RoomAuthorizedData}) holding access pins and aliases, and the history of the reservation request (\texttt{List<ReservationRequestHistoryModel>}) holding information about how the request changed over the time.
        \item \textbf{[PATCH]} -- Modifies the reservation request with parameters specified by \texttt{ReservationRequestCreateModel} acquired from the request body.
        \item \textbf{[DELETE]} -- Deletes the reservation request.
    \end{description}
    \item \textbf{[POST] /api/v1/reservation\_requests/\{id:.+\}/accept} -- The resource owner or administrator can use this endpoint to accept the reservation request awaiting confirmation, so it becomes ready for allocation and eventually a reservation.
    \item \textbf{[POST] /api/v1/reservation\_requests/\{id:.+\}/reject} -- The resource owner or administrator can use this endpoint to reject the reservation request awaiting confirmation.
    \item \textbf{[POST] /api/v1/reservation\_requests/\{id:.+\}/reject} -- Reverts the modifications of the reservation request.
\end{itemize}


\begin{table}[H]
    \label{tab:res}
    \begin{tabularx}{\textwidth}{llX}
        \toprule
        Name & Type & Description \\
        \midrule
        resource & QUERY & Filters entries that use given \emph{resource} \\
        type & QUERY & Filters entries with given \emph{type} \\
        search & QUERY & Filters entries that contains \emph{search} \\
        participant\_user\_id & QUERY &  Filters entries that have participant with id \emph{participant\_user\_id} \\
        user\_id & QUERY & Filters entries that owner's id equals \emph{user\_id} \\
        interval\_from & QUERY & Filters entries from this date \\  
        interval\_to & QUERY & Filters entries to this date \\
        technology & QUERY & Filters entries by given \emph{TechnologyModel} \\
        parentRequestId & QUERY & Filters entries that have \emph{parentRequestId} \\
        allocation\_state & QUERY & Filters entries that have \emph{allocation\_state} \\
        \bottomrule
    \end{tabularx}
    \caption{List reservation requests parameters table.}
\end{table}
