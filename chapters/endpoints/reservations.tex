\section{Reservation Requests}
\subsection{List Reservation Requests}
\begin{table}[ht!]
    \begin{tabularx}{\textwidth}{llX}
        \toprule
        Name & Type & Description \\
        \midrule
        resource & QUERY & Filters entries that use given \emph{resource} \\
        type & QUERY & Filters entries with given \emph{type} \\
        search & QUERY & Filters entries that contains \emph{search} \\
        participant\_user\_id & QUERY &  Filters entries that have participant with id \emph{participant\_user\_id} \\
        user\_id & QUERY & Filters entries that owner's id equals \emph{user\_id} \\
        interval\_from & QUERY & Filters entries from this date \\  
        interval\_to & QUERY & Filters entries to this date \\
        technology & QUERY & Filters entries by given \emph{TechnologyModel} \\
        parentRequestId & QUERY & Filters entries that have \emph{parentRequestId} \\
        allocation\_state & QUERY & Filters entries that have \emph{allocation\_state} \\
        sort & QUERY & Sort entries by \emph{sort} parameter \\
        sort-desc & QUERY & Sorts entries in descending order if true \\
        start & QUERY & \emph{Start} the listing from this number \\  
        count & QUERY & \emph{Count} of records to return
        \end{tabularx}
    \caption{List roles parameters table.}
\end{table}
\begin{description}
    \item \textbf{HTTP request}\\
        \texttt{\text{[GET]} /api/v1/reservation\_requests}
    \item \textbf{Description}\\
        Lists \texttt{ReservationRequest}s.
    \item \textbf{Response}\\
        \texttt{\text{[200 OK]} ListResponse<ReservationRequest>}
\end{description}

\subsection{Create Reservation Requests}
\begin{description}
    \item \textbf{HTTP request}\\
        \texttt{\text{[POST]} /api/v1/reservation\_requests \texttt{ReservationRequest}}
    \item \textbf{Description}\\
        Creates \texttt{ReservationRequest} given in request body.
    \item \textbf{Response}\\
        \texttt{\text{[200 OK]}}
\end{description}

\subsection{Delete Reservation Requests}
\begin{description}
    \item \textbf{HTTP request}\\
        \texttt{\text{[DELETE]} /api/v1/reservation\_requests/\{id\} \texttt{ReservationRequest}}
    \item \textbf{Description}\\
        Creates \texttt{ReservationRequest} given in request body.
    \item \textbf{Response}\\
        \texttt{\text{[200 OK]}}
\end{description}
