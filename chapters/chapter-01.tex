\chapter{Using lightweight markup}
\shorthandoff{-}
\begin{markdown*}{%
  hybrid,
  definitionLists,
  footnotes,
  inlineFootnotes,
  hashEnumerators,
  fencedCode,
  citations,
  citationNbsps,
  pipeTables,
  tableCaptions,
}

If you decide that \LaTeX{} is too wordy for some parts of your
document, there are [packages](https://www.ctan.org/pkg/markdown
"Markdown") that allow you to use more lightweight markup next
to it.

 ![logo](fithesis/logo/mu/fithesis-fi-color "The old-style seal of
         the Faculty of Informatics")

| Right | Left | Default | Center |
|------:|:-----|---------|:------:|
|    12 | 12   | 12      |   12   |
|   123 | 123  | 123     |   123  |
|     1 | 1    | 1       |    1   |

: This is a table with different types of alignment.

This is a bullet list. Unlike numbered lists, bulleted lists
contain an **unordered** set of bullet points. When a bullet point
contains multiple paragraphs, the list is typeset as follows:

  * The first item of a bullet list

    that spans several paragraphs,
  * the second item of a bullet list,
  * the third item of a bullet list.

When none of the bullet points contains multiple paragraphs, the
list has a more compact form:

  * The first item of a bullet list,
  * the second item of a bullet list,
  * the third item of a bullet list.

Unlike a bulleted list, a numbered list implies chronology or
ordering of the bullet points. When a bullet point
contains multiple paragraphs, the list is typeset as follows:

  1. The first item of an ordered list

     that spans several paragraphs,
  2. the second item of an ordered list,
  3. the third item of an ordered list.
  #. If you are feeling lazy,
  #. you can use hash enumerators as well.

When none of the bullet points contains multiple paragraphs, the
list has a more compact form:

  6. The first item of an ordered list,
  7. the second item of an ordered list,
  8. the third item of an ordered list.

Definition lists are used to provide definitions of terms. When
a definition contains multiple paragraphs, the list is typeset
as follows:

Term 1

:   Definition 1

*Term 2*

:   Definition 2

        Some code, part of Definition 2

    Third paragraph of Definition 2.

When none of the bullet points contains multiple paragraphs, the
list has a more compact form:

Term 1
:   Definition 1
*Term 2*
:   Definition 2

Block quotations are used to include an excerpt from an external
document in way that visually clearly separates the excerpt from
the rest of the work:

> This is the first level of quoting.
>
> > This is nested blockquote.
>
> Back to the first level.

Footnotes are used to include additional information to the
document that are not necessary for the understanding of the main
text. Here is a footnote reference^[Here is the footnote.] and
another.[^longnote]

[^longnote]: Here's one with multiple blocks.

    Subsequent paragraphs are indented to show that they
belong to the previous footnote.

        Some code

    The whole paragraph can be indented, or just the first
    line.  In this way, multi-paragraph footnotes work like
    multi-paragraph list items.

Citations are used to provide bibliographical references to other
documents. This is a regular citation~[@borgman03, p. 123]. This is
an in-text citation: @borgman03\. You can also cite several authors
at once using both regular~[see @borgman03, p. 123; @greenberg98,
sec.  3.2; and @thanh01] and in-text citations: @borgman03 [p.123;
@greenberg98, sec. 3.2; @thanh01].

Code blocks are used to include source code listings into the
document:

    #include <stdio.h>
    #include <unistd.h>
    #include <sys/types.h>
    #include <sys/wait.h>
    // This is a comment
    int main(int argc, char **argv)
    {
        while (--c > 1 && !fork());
        sleep(c = atoi(v[c]));
        printf("%d\n", c);
        wait(0);
        return 0;
    }

There is an alternative syntax for code blocks that allows you to
specify additional information, such as the language of the source
code. This information can be used for syntax highlighting:

~~~~~~ Ruby
# Here's a way to empty an array.
joe = [ 'eggs.', 'some', 'break', 'to', 'Have' ]
print(joe.pop, " ") while joe.size > 0
print "\n"
~~~~~~

\end{markdown*}
\shorthandon{-}
