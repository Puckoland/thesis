\chapter{Reservation System Shongo}
This chapter introduces the subject of this work, the reservation system Shongo.
Information for this chapter was drawn mainly from the Shongo API documentation \footfullcite{shongoapi}. However, this documentation is outdated and is only partially correct. Since the documentation had been introduced, multiple theses and other works have contributed to the development of the Shongo system. Thus, much information has been drawn from these, especially the theses concerning inter-domain extension \footfullcite{pavelka2016shongo} and redesign of the system \footfullcite{perichta2020}.

Currently, at version v0.9.3, Shongo is a generic system used for managing resources.
Its primary function is to manage resource reservations for virtual meetings and any physical resources, such as physical meeting rooms, vehicles or parking places.
It allows administrators to define available resources (e.g., H.323 MCU or Adobe Connect Server) and users to request a reservation for the available resources.
Furthermore, the system provides remote control and recording of ongoing virtual rooms using the Cisco TCS recording server, user notifications, and detailed resource usage statistics.
\cite{shongo}

The system is developed as an open-source project available at \href{https://github.com/shongo/shongo}{Github repository} by CESNET.
CESNET is an association of universities and the Academy of Sciences of the Czech Republic. The association offers a wide range of services and develops a national electronic infrastructure for science and education. The e-infrastructure includes a computer network, computing grids, data repositories, and collaborative environments. \cite{cesnet}

The system is currently deployed in these domains:
\begin{itemize}
    \item \url{https://meetings.cesnet.cz/} -- reservation system for \\
    H.323/SIP and Adobe Connect virtual rooms within CESNET videoconferencing infrastructure
    \item \url{https://meetings.cesnet.cz/ceitec/} -- reservation system for physical rooms of CEITEC --- Central European Institute of Technology
    \item \url{https://meetings.cesnet.cz/cuni/} -- reservation system for users led in the authentication system of Charles University\footnote{\url{https://cuni.cz/}}.
\end{itemize}

Users can log in using their profile from any organization belonging to the Czech academic identity federation \footfullcite{eduid}.
After logging in, they can freely use the system and request reservations for available resources.

\section{Reservations}
As already mentioned, the primary task of the Shongo system is to manage reservations. This section describes the fundamental concepts behind the reservation logic.

Physical reservations are very straightforward:
\begin{enumerate}
    \item A user requests a reservation of a physical resource.
    \item If the request is permitted, he can use the physical resource for a specified period.
\end{enumerate}

On the other hand, virtual reservations are a little bit more complicated. Virtual rooms represent the allocation of a specific virtual resource. There are two types of virtual rooms:
\begin{itemize}
    \item \textbf{Permanent rooms} are devoted to long-term reservations. This type reserves virtual resource --- name and a unique identifier of room (URL, phone number, SIP URI) --- but without capacity reservation for the chosen resource (period, periodicity, number of participants). The user has to book capacity separately after creating the permanent room.
    \item \textbf{One-time (ad-hoc) rooms} are intended for one-time reservations of virtual resources. The controller generates a name and unique identifier according to available values from the range configured in the resource specification. That results in a new room name and identification for each reservation. A capacity is reserved when the room is created, and the room is deleted after the requested time slot ends.
\end{itemize}
After reserving the room, the user can set up roles for other users or groups (owner, user, reader) and add other participants with specific permissions (administrator, presenter, participant).
If the reservation is permitted (sufficient capacity is available), the user who created the reservation and the participants added by the user can join the videoconference during the requested time slot.

\section{Architecture}
Explaining what parts of the system were added, modified, or replaced without fundamental knowledge of the system architecture would be confusing and perhaps impossible to comprehend. This section explains the system’s architecture and components with a closer look.

The system's Architecture is composed of a few separate modules. Their interconnections are shown in \Cref{fig:architecture}.

\subsection{Controller} \label{controller}
The controller module is the heart of the Shongo system.
Its primary function is to plan requested reservations (\emph{Scheduler} sub-component) and allocate necessary resources.
Its secondary roles are informing users about their reservations (\emph{NotificationManager} sub-component) and administrating remote virtual resources (\emph{Executor} sub-component) through connected \hyperref[connector]{connectors}.
Furthermore, last but not least, it is responsible for saving this information into the relational database \emph{PostgreSQL}.

These functionalities are made available for other components with the \emph{API} (Application programming interface) via protocol \emph{XML-RPC}.

\subsection{Connector} \label{connector}
The connector is a component that enables users to connect to video rooms with their devices (PC, tablet, smartphone).
It is responsible for establishing, maintaining and monitoring connections to remote virtual resources.
It contains implementations for \emph{Device Agent}s, and each \emph{Device Agent} provides a specific device (e.g. Adobe Connect, Pexip) features to the controller using the device’s \emph{API}.
Moreover, it enables the system to save data from the device to the file system.

The controller communicates with the connector using the \emph{JADE} (Java Agent DEvelopment) framework, which enables the connection of multiple connectors to a single controller. This can be done, for example, in order to distribute the load.

\subsection{Authentication Server}
The authentication server authenticates incoming requests for the controller component, authenticates users, finds users from various identity providers and obtains information about users for the web client.
Besides that, it also manages the authentication layer of the web conference system Adobe Connect. This layer uses the user's information gained after logging in via system Shibboleth\footnote{Shibboleth, \url{https://www.shibboleth.net/}}.

\subsection{Web Client} \label{webclient}
The web client is the system's primary interface for users. The \emph{SpringMVC} and \emph{AngularJS} frameworks generate web pages with data gained from the controller’s services.

Thanks to this component, users can authenticate via OpenID Connect \footnote{OpenID Connect, \url{https://openid.net/connect/}} and request a reservation of any available resource comfortably via the web browser.

\subsection{Command-line Client}
The command-line client is the controller's additional interface used for administration. It allows system administrators to manage resources, users, groups and reservation requests which is not possible using the web client.

\scalegraphics{assets/shongo_architecture}{Shongo architecture}{architecture}
