\begin{lstlisting}[language=java, caption=ReservationRequestController.java, label=lst:controller]
// Makes this path a Controller (auto-detectable) and binds return values of methods to the HTTP request body
@RestController
// Binds this controller to the path parameter
@RequestMapping("/api/v1/reservation_requests")
public class ReservationRequestController
{
    // Controller service used to manage reservations
    private final ReservationService reservationService;
    
    // @Autowired resolves the service to Bean defined in rest-api-servlet.xml
    public ReservationRequestController(@Autowired ReservationService reservationService)
    {
        this.reservationService = reservationService;
    }
    
    // Binds this method to the HTTP GET request with the path defined in @RequestMapping
    @GetMapping
    // Method returns ListResponse<ReservationRequestModel> in response body thanks to @RestController annotation
    ListResponse<ReservationRequestModel> listRequests(
        // Gets security token from request attributes. The token was stored there when AuthFilter processed the request.
        @RequestAttribute(TOKEN) SecurityToken securityToken,
        // Reads optional request parameter allocation_state and stores it in allocationState variable
        @RequestParam(value = "allocation_state", required = false) AllocationState allocationState,
        ...)
    {
        // Use Controller service to acquire requested data
        ReservationRequestListRequest request = new ReservationRequestListRequest();
        request.setAllocationState(allocationState);
        ...
        ListResponse<ReservationRequestSummary> response = reservationService.listReservationRequests(request);
        ...
        // Return the acquired data
        return response;
    }
    
    // Sets the response status for successful execution of the method
    @ResponseStatus(HttpStatus.CREATED)
    // Binds this method to the HTTP POST request with the path defined in @RequestMapping
    @PostMapping
    void createRequest(
        @RequestAttribute(TOKEN) SecurityToken securityToken,
        // Reads the request body and deserializes it to the ReservationRequest object
        @RequestBody ReservationRequest request)
    {
        ...
        // Use Controller service to create reservation according to the requested data
        reservationService.createReservationRequest(
            securityToken, request.toApi());
    }
    
    // Sets possible HTTP responses
    @ApiResponses(value = {
        @ApiResponse(responseCode = "200"), @ApiResponse(responseCode = "404", description = "Reservation request not found.", content = @Content),
    })
    // Binds this method to the HTTP DELETE request with the path defined in @RequestMapping concatenated with "/{id:.+}". The ':' marks the start of a regex used for the id variable (useful to accept only numbers, for example). The regex ".+" is used because shongo-id can include URL key characters like ':' or '-'.
    @DeleteMapping("/{id:.+}")
    void deleteRequest(
        @RequestAttribute(TOKEN) SecurityToken securityToken,
        // Reads the id variable defined in request path
        @PathVariable String id)
    {
        // Use Controller service to delete reservation with id defined in path
        reservationService.deleteReservationRequest(
            securityToken, id);
    }
    ...
}
\end{lstlisting}